\input{../preamble}

%----------------------------------------------------------------------------------------
%	TITLE PAGE
%----------------------------------------------------------------------------------------

\title[第1讲]{第2讲 :OS Architecture \& Structure} % The short title appears at the bottom of every slide, the full title is only on the title page
\subtitle{第三节:Monolithic kernel -- UNIX }
\author{陈渝} % Your name
\institute[清华大学] % Your institution as it will appear on the bottom of every slide, may be shorthand to save space
{
	清华大学计算机系 \\ % Your institution for the title page
	\medskip
	\textit{yuchen@tsinghua.edu.cn} % Your email address
}
\date{\today} % Date, can be changed to a custom date


\begin{document}

\begin{frame}
\titlepage % Print the title page as the first slide
\end{frame}

%\begin{frame}
%\frametitle{提纲} % Table of contents slide, comment this block out to remove it
%\tableofcontents % Throughout your presentation, if you choose to use \section{} and \subsection{} commands, these will automatically be printed on this slide as an overview of your presentation
%\end{frame}
%
%%----------------------------------------------------------------------------------------
%%	PRESENTATION SLIDES
%%----------------------------------------------------------------------------------------
%
%%------------------------------------------------
%\section{第一节:课程概述} % Sections can be created in order to organize your presentation into discrete blocks, all sections and subsections are automatically printed in the table of contents as an overview of the talk
%%------------------------------------------------
%-------------------------------------------------
\begin{frame}[plain]
	\frametitle{UNIX}
	
	
	
	\begin{columns}
		
		\begin{column}{.4\textwidth}
			
			\includegraphics[width=1.\textwidth]{unix-authors}
			
		\end{column}
		
		\begin{column}{.6\textwidth}
			
%			\Large
			UNIX
			\begin{itemize}
				\item Ken Thompson, Dennis Ritchie
				\item Douglas McIlroy, and J. F. Ossanna
		 		\item  The name (written Unics at the beginning) was coined by Brian
		 		Kernighan as a pun on Multics
		 	
			\end{itemize}	
			
%			\includegraphics[width=1.\textwidth]{msdos}		
		\end{column}
		
		
	\end{columns}
	
	
\end{frame}

%-------------------------------------------------
\begin{frame}[plain]
	\frametitle{UNIX}
	

		\begin{itemize}
		\item Thompson wrote the first version of the yet-unnamed operating
		system in assembly language for a DEC PDP-7 minicomputer
		\item  Thompson developed a compiler for a new high-level language
		he called B (stripped-down version of the BCPL language).  
		In 1972 Dennis Ritchie created a new language called C
		\item In 1973 most of the UNIX kernel was
		rewritten in C
						
		\end{itemize}
			\centering
			\includegraphics[width=.7\textwidth]{Thompson}
	
\end{frame}

%-------------------------------------------------
\begin{frame}[plain]
	\frametitle{UNIX}
	
	\centering
	\includegraphics[width=1.\textwidth]{unix-family}
	
\end{frame}


%-------------------------------------------------
\begin{frame}[plain]
	\frametitle{UNIX architecture}
	
	\centering
	\includegraphics[width=.9\textwidth]{unix-arch}
	
\end{frame}

%-------------------------------------------------
\begin{frame}[plain]
	\frametitle{UNIX architecture}
	
	\centering
	\includegraphics[width=1.\textwidth]{unix-fs-1}
	
\end{frame}

%-------------------------------------------------
\begin{frame}[plain]
	\frametitle{UNIX architecture}
	
	\centering
	\includegraphics[width=1.\textwidth]{unix-fs-2}
	
\end{frame}

%-------------------------------------------------
\begin{frame}[plain]
	\frametitle{UNIX architecture}
	
	\centering
	\includegraphics[width=1.\textwidth]{unix-fs-3}
	
\end{frame}


%-------------------------------------------------
\begin{frame}[plain]
	\frametitle{Linux architecture}
	
	\centering
	\includegraphics[width=1.\textwidth]{linux-arch}
	
\end{frame}
%-------------------------------------------------


\end{document}