\input{../preamble}

%----------------------------------------------------------------------------------------
%	TITLE PAGE
%----------------------------------------------------------------------------------------

\title[第3讲]{第3讲 :Virtual Machine Monitor} % The short title appears at the bottom of every slide, the full title is only on the title page
\subtitle{第三节:Hardware-assisted Virtualization}
\author{陈渝} % Your name
\institute[清华大学] % Your institution as it will appear on the bottom of every slide, may be shorthand to save space
{
	清华大学计算机系 \\ % Your institution for the title page
	\medskip
	\textit{yuchen@tsinghua.edu.cn} % Your email address
}
\date{\today} % Date, can be changed to a custom date


\begin{document}

\begin{frame}
\titlepage % Print the title page as the first slide
\end{frame}

%\begin{frame}
%\frametitle{提纲} % Table of contents slide, comment this block out to remove it
%\tableofcontents % Throughout your presentation, if you choose to use \section{} and \subsection{} commands, these will automatically be printed on this slide as an overview of your presentation
%\end{frame}
%
%%----------------------------------------------------------------------------------------
%%	PRESENTATION SLIDES
%%----------------------------------------------------------------------------------------
%
%%------------------------------------------------
%\section{第一节:课程概述} % Sections can be created in order to organize your presentation into discrete blocks, all sections and subsections are automatically printed in the table of contents as an overview of the talk
%%------------------------------------------------

%-------------------------------------------------
\begin{frame}[plain]
	\frametitle{Problems }
	
	
	
	\begin{columns}
		
		\begin{column}{.3\textwidth}
			
			\includegraphics[width=1.\textwidth]{vmm-overview}
			
		\end{column}
		
		\begin{column}{.7\textwidth}
			
			\textbf{Virtualization holes}

		 Conventional Intel® 64, or IA-32, is not virtualizable architecture
			\begin{itemize}
				\item Sensitive register instructions: read or change sensitive registers and/or memory locations such as a clock register or interrupt registers

					\begin{itemize}
					\item  SGDT, SIDT, SLDT
					\item  SMSW 
					\item PUSHF, POPF 
					
					\end{itemize} 
				\item Protection system instructions: reference the storage protection system, memory or address relocation system
			
					\begin{itemize}
					\item  LAR, LSL, VERR, VERW 
					\item  POP, PUSH
					\item  CALL, JMP, INT n, RET 
					\item STR, MOVE
					\end{itemize} 
		
			\end{itemize} 
				\tiny Source from proceeding of 2000 USENIX ATC
		\end{column}
		
		
	\end{columns}
	
	
\end{frame}


%-------------------------------------------------
\begin{frame}[plain]
	\frametitle{Problems}
	
	
	
	\begin{columns}
		
		\begin{column}{.3\textwidth}
			
			\includegraphics[width=1.\textwidth]{vmm-overview}
			
		\end{column}
		
		\begin{column}{.7\textwidth}
			
			\textbf{Controlling the CPU Resource}
			
			With an Intel® 64 CPU, a VMM must be able to retain control over
			\begin{itemize}
				\item Access to privileged state (CRn, DRn, MSRs)
				\item Exceptions (\#PF, \#MC, etc.)
				\item Interrupts and interrupt masking
				\item Address translation (via page tables)
				\item CPU access to I/O (via I/O ports or MMIO)
			\end{itemize} 

		\end{column}
		
		
	\end{columns}
	
	
\end{frame}

%-------------------------------------------------
\begin{frame}[plain]
	\frametitle{Problems }
	
	
	
	\begin{columns}
		
		\begin{column}{.3\textwidth}
			
			\includegraphics[width=1.\textwidth]{vmm-overview}
			
		\end{column}
		
		\begin{column}{.7\textwidth}
			
			\textbf{CPU Control via “Ring Deprivileging”}
			
			With an Intel® 64 CPU, a VMM must be able to retain control over
			\begin{itemize}
				\item Guest OS kernel runs in a less privileged ring than usual
				
				\item VMM runs in the most privileged ring 0
				\item prevent guest OS from accessing privileged instructions / state
				\item prevent guest OS from modifying VMM code and data

			\end{itemize} 
			
		\end{column}
		
		
	\end{columns}
	
	
\end{frame}


%-------------------------------------------------
\begin{frame}[plain]
	\frametitle{Problems }
	
	
	
	\begin{columns}
		
		\begin{column}{.3\textwidth}
			
			\includegraphics[width=1.\textwidth]{vmm-overview}
			
		\end{column}
		
		\begin{column}{.7\textwidth}
			
			\textbf{Problems with Ring Deprivileging}
			
			Ring deprivileging encounters various issues in current Intel® 64 architecture:
			\begin{itemize}
				\item Ring compression/aliasing
				\item Non-faulting reads of privileged state
				\item Excessive faulting
				\item Interrupt-virtualization issues
				
			\end{itemize} 
			
		\end{column}
		
		
	\end{columns}
	
	
\end{frame}


%-------------------------------------------------
\begin{frame}[plain]
	\frametitle{Solutions }
	
	
	
	\begin{columns}
		
		\begin{column}{.3\textwidth}
			
			\includegraphics[width=1.\textwidth]{vmm-overview}
			
		\end{column}
		
		\begin{column}{.7\textwidth}
			
			\textbf{Addressing  “Virtualization Holes”}
			
			\begin{itemize}
				\item Paravirtualization
				\item Binary translation (or patching)
				\item Hardware-assisted virtualization
				\begin{itemize}
								
				\item Closing virtualization holes in hardware
				\item Simplify VMM software
				\item Optimizing for performance
				\end{itemize} 
			\end{itemize} 
			
		\end{column}
		
		
	\end{columns}
	
	
\end{frame}


%-------------------------------------------------
\begin{frame}[plain]
	\frametitle{Solution -- hardware-assisted VT }
	
	
	
	\begin{columns}
		
		\begin{column}{.3\textwidth}
			
			\includegraphics[width=1.\textwidth]{vmm-overview}
			
		\end{column}
		
		\begin{column}{.7\textwidth}
			
			\textbf{Intel® Virtualization Technology}
			
			A hardware-assisted virtualization technology, named as Intel® Virtualization Technology, or Intel® VT 
			\begin{itemize}
				\item For Intel® 64, VT-x: CPU MEM
				\item For directed I/O, VT-d: DMA/Interrupt remapping
				\item For connectivity, VT-c: offload wor from CPU to IO device

			\end{itemize} 
			
		\end{column}
		
		
	\end{columns}
	
	
\end{frame}

%-------------------------------------------------
\begin{frame}[plain]
	\frametitle{VT-x -- cpu}
	
	
	
	\begin{columns}
		
		\begin{column}{.5\textwidth}
			
			\includegraphics[width=.8\textwidth]{vmx-overview}
			
		\end{column}
		
		\begin{column}{.5\textwidth}
			
			\textbf{VM entry}
			
			\begin{itemize}
				\item Transition from VMM to guest
				\item Enters VMX non-root operation
				\item VMLAUNCH used for initial entry
				\item VMRESUME used subsequently
				
			\end{itemize} 
			
			\textbf{VM exit}
			
			\begin{itemize}
				\item Guest-to-VMM transition
				\item Enters VMX root operation
				\item Caused by external events,
				exceptions, some instructions

				
			\end{itemize} 
			
		\end{column}
		
		
	\end{columns}
	
	
\end{frame}

%-------------------------------------------------
\begin{frame}[plain]
	\frametitle{VT-x -- cpu}
	
	
	
	\begin{columns}
		
		\begin{column}{.5\textwidth}
			
			\includegraphics[width=1.\textwidth]{vt-x}
			
		\end{column}
		
		\begin{column}{.5\textwidth}
			
			\textbf{Virtual Machine Control Structure (VMCS)}
			
%			\hline
			Controls VMX non-root operation/transitions
			\begin{itemize}
				\item Each virtual CPU should have its own VMCS
				\item Only one VMCS active at a time on a physical CPU
				
			\end{itemize} 
			
			Each VMCS stored in a memory region
			
			\begin{itemize}
				\item Physical address set by new VMPTRLD instruction
				\item Need not reside in linear-address space 
				\item VMPTRLD also used to switch VMCS
				
				
			\end{itemize}
			
		\end{column}
		
		
	\end{columns}
	
	
\end{frame}

%-------------------------------------------------
\begin{frame}[plain]
	\frametitle{VT-x -- mem}
	
	
	
	\begin{columns}
		
		\begin{column}{.5\textwidth}
			
			\includegraphics[width=1.\textwidth]{vt-x-mem}
			
		\end{column}
		
		\begin{column}{.5\textwidth}
			
			\textbf{Memory virtualization challenges}
			
%			\hline
			
			\begin{itemize}
				\item OS expect to see physical memory starting from 0
				\item OS expect to see contiguous memory in address space
				\item BIOS/Legacy OS are designed to boot from address low 1M
				\item DMA, TLB, etc.
			\end{itemize} 
			
			
			
		\end{column}
		
		
	\end{columns}
	
	
\end{frame}

%-------------------------------------------------
\begin{frame}[plain]
	\frametitle{VT-x -- mem}
	
			
			\includegraphics[width=1.\textwidth]{vt-x-ept}
		

			\textbf{Memory virtualization challenges}
			
			

			\begin{itemize}
				\item Guest can have full control over its page tables and events: 
				\begin{itemize}
				\item 	CR3, INVLPG, page fault
				\end{itemize}
				\item VMM controls Extended Page Tables:
				\begin{itemize}
				\item BIOS/Legacy OS are designed to boot from address low 1M
				\item DMA, TLB, etc.
				\end{itemize}
			\end{itemize} 
		
	
\end{frame}

%-------------------------------------------------
\begin{frame}[plain]
	\frametitle{VT-d}
	

	\begin{columns}
	
	\begin{column}{.4\textwidth}
		
		\includegraphics[width=1.\textwidth]{vt-d-direct}
		
	\end{column}
	
	\begin{column}{.5\textwidth}
		
	\textbf{Direct assignment}
	
	
	
	\begin{itemize}
		\item Guest runs native driver  
		\item I/O is written through
		\item Physical interrupt is captured by hypervisor (pIRQ)
		\item Virtual interrupt is signaled for guest (vIRQ)
		\item Remapping from pIRQ <-> vIRQ in hypervisor
	\end{itemize} 
	
	\textbf{Maximizing performance, but sacrificing sharing}
	
    \end{column}


\end{columns}
	
\end{frame}


%-------------------------------------------------
\begin{frame}[plain]
	\frametitle{VT-d}
	
	\centering
	\includegraphics[width=.8\textwidth]{vt-d-overview}
	
	
	\textbf{Memory virtualization challenges}
	
	
	
\end{frame}

%-------------------------------------------------
\begin{frame}[plain]
	\frametitle{VT-c}
	
	
	\begin{columns}
		
		\begin{column}{.5\textwidth}
			
			\includegraphics[width=.75\textwidth]{sriov}
			
		\end{column}
		
		\begin{column}{.5\textwidth}
			
			\textbf{Single Root I/O Virtualization (SR-IOV)}
			
			
			
			\begin{itemize}
				\item Close to native performance  
				\item Limited CPU overhead
				\item Flexible sharing
				\item Security control through PF
			\end{itemize} 
			
			\textbf{Maximizing performance}
			
		\end{column}
		
		
	\end{columns}
	
\end{frame}
%-------------------------------------------------
%-------------------------------------------------
\begin{frame}[plain]
	\frametitle{Summary}
	\centering
	
\includegraphics[width=.9\textwidth]{rv-arm-1}
	
\end{frame}

%-------------------------------------------------
\begin{frame}[plain]
	\frametitle{Summary}
	\centering
	
	\includegraphics[width=.9\textwidth]{rv-arm-2}
	
\end{frame}


%-------------------------------------------------
\begin{frame}[plain]
	\frametitle{Summary}
	\centering
	
	\includegraphics[width=.9\textwidth]{rv-kvm-linux}
	
\end{frame}
%-------------------------------------------------
%-------------------------------------------------


\end{document}