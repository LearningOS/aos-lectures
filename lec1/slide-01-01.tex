\input{../preamble}

%----------------------------------------------------------------------------------------
%	TITLE PAGE
%----------------------------------------------------------------------------------------

\title[第1讲]{第1讲 :Advanced OS Overview} % The short title appears at the bottom of every slide, the full title is only on the title page
\subtitle{第一节:Course Overview }
\author{陈渝} % Your name
\institute[清华大学] % Your institution as it will appear on the bottom of every slide, may be shorthand to save space
{
	清华大学计算机系 \\ % Your institution for the title page
	\medskip
	\textit{yuchen@tsinghua.edu.cn} % Your email address
}
\date{\today} % Date, can be changed to a custom date


\begin{document}

\begin{frame}
\titlepage % Print the title page as the first slide
\end{frame}

%\begin{frame}
%\frametitle{提纲} % Table of contents slide, comment this block out to remove it
%\tableofcontents % Throughout your presentation, if you choose to use \section{} and \subsection{} commands, these will automatically be printed on this slide as an overview of your presentation
%\end{frame}
%
%%----------------------------------------------------------------------------------------
%%	PRESENTATION SLIDES
%%----------------------------------------------------------------------------------------
%
%%------------------------------------------------
%\section{第一节:课程概述} % Sections can be created in order to organize your presentation into discrete blocks, all sections and subsections are automatically printed in the table of contents as an overview of the talk
%%------------------------------------------------
%-------------------------------------------------
\begin{frame}[plain,t]
	\frametitle{课程信息}
	\begin{itemize}
		\item Instructor:陈渝	 
		\item Research area: OS
	\end{itemize}	
	\includegraphics[width=0.2\linewidth]{chenyu}

 	\begin{itemize}
 		\item TA: ...
		\item Course Representatives:王润基、贾越凯、戴臻旸、王逸松...
	\end{itemize}


\end{frame}
%-------------------------------------------------
\begin{frame}[plain]

\frametitle{预备知识}

\begin{itemize}

\item 程序设计语言(汇编、C/C++、Go、Rust)
\item 数据结构和算法 \pause
\item 编译原理/操作系统 
\item 计算机组成原理/计算机体系结构 \pause

\item English
\end{itemize}

\end{frame}
%-------------------------------------------------
\begin{frame}[plain]

\frametitle{Why Study OS?}

\begin{itemize}
	
	\item The Operating System (OS) I use has already been written, and I doubt it will be my job to write another one. For example, Windows, Linux. \pause
	\item Haven't OS developers figured everything out already? What more is there to do? \pause
	\item Why should I study this as a graduate student?

\end{itemize}
	\centering
	\includegraphics[width=0.6\linewidth]{why-study-os}
\end{frame}
%-------------------------------------------------
\begin{frame}[plain]	
	\frametitle{Objectives}

\begin{itemize}\Large 
	\item Gain experience in doing OS research
	
	\begin{itemize}\large 
		\item Know how to read/write papers/reports
		\item Know current OS hot topics
		\item Develop OS projects \pause
		\item Help other CS researches
	\end{itemize}
\end{itemize}
	\centering
	\includegraphics[width=0.45\linewidth]{system-for-ai}
\end{frame}

%-------------------------------------------------
\begin{frame}[plain]	
	\frametitle{Course Materials}
	
	\begin{itemize}\Large 
		\item Lecture notes/papers (14 research domains)
		
		\begin{itemize}\large 
			\item  \href{https://github.com/chyyuu/aos\_course/blob/master/readinglist.md}{ReadingList of OS}
			\begin{itemize}\large 
				\item OS Arch,  Process/Thread/Scheduling
				\item Memory Management, Concurrency/Sync/Mutex
				\item Distributed Systems, Virtual Machine Monitor
				\item Network, File System, Scalability
				\item Bugs/Security/Fault-Tolerant/Recovery
				\item Encryption Authentication
				\item Interface Design, Verification/Proof, DEVICES
			\end{itemize}
		\end{itemize}
	\end{itemize}
	
\end{frame}

%-------------------------------------------------
\begin{frame}[plain]	
	\frametitle{Course Materials}
	\Large 
	\textbf{Reference Books}
	\begin{itemize}\large 
		\item Wolfgang Mauerer, \textbf{Professional Linux Kernel Architecture}
		\item Uresh Vahalia, \textbf{UNIX Internals-- The New Frontiers}
		\item Daniel P. Bovet, \textbf{Understanding the Linux Kernel}
		\item Mark E. Russinovich,\textbf{ Microsoft Windows Internals}
		\item Tanenbaum, \textbf{Modern Operating Systems}

	\end{itemize}
	
\end{frame}

%-------------------------------------------------


\end{document}