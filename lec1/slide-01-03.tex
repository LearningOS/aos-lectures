\input{../preamble}

%----------------------------------------------------------------------------------------
%	TITLE PAGE
%----------------------------------------------------------------------------------------

\title[第1讲]{第1讲 :Advanced OS Overview} % The short title appears at the bottom of every slide, the full title is only on the title page
\subtitle{第三节:Rethink OS Components}
\author{陈渝} % Your name
\institute[清华大学] % Your institution as it will appear on the bottom of every slide, may be shorthand to save space
{
	清华大学计算机系 \\ % Your institution for the title page
	\medskip
	\textit{yuchen@tsinghua.edu.cn} % Your email address
}
\date{\today} % Date, can be changed to a custom date


\begin{document}

\begin{frame}
\titlepage % Print the title page as the first slide
\end{frame}

%\begin{frame}
%\frametitle{提纲} % Table of contents slide, comment this block out to remove it
%\tableofcontents % Throughout your presentation, if you choose to use \section{} and \subsection{} commands, these will automatically be printed on this slide as an overview of your presentation
%\end{frame}
%
%%----------------------------------------------------------------------------------------
%%	PRESENTATION SLIDES
%%----------------------------------------------------------------------------------------
%
%%------------------------------------------------
%\section{第一节:课程概述} % Sections can be created in order to organize your presentation into discrete blocks, all sections and subsections are automatically printed in the table of contents as an overview of the talk
%%------------------------------------------------
%-------------------------------------------------
\begin{frame}
	\frametitle{OS Overview}

\Large
\centering
History of OS: Change!
\large
\begin{table}[]
	\begin{tabular}{|l|l|l|l|l|}
		\hline
		Porperty & Element             & 1980        & 2020          & Factor       \\ \hline
		Speed    & CPU            & 1 		MIPS    & 97,000 		MIPS & $9.7 \times  10^5 $  \\ \hline
		& Memory         & 500 		ns    & 0.4 		ns      & $ 3.6 \times  10^3 $  \\ \hline
		& Disk           & 18 		ms     & 4 		ms        & $ 2.2 \times   10^1  $ \\ \hline
		& Network        & 300 		b/sec & 100 		Gb/sec  & $ 3.3	\times   10^8 $  \\ \hline
		Capacity & Memory         & 64 		KB     & 1 		TB       & $ 2.5 \times   10^5 $  \\ \hline
		& Disk           & 1 		MB      & 4 		TB        & $ 4.0 \times   10^6 $  \\ \hline
		Other    & Address 		bits & 8           & 64            & $ 8  $           \\ \hline
		& Users/machine  & 10          & 0.1           & $ 1.0 		\times  10^{-2}$  \\ \hline
		Cost     & Per 		MIP      & \$100K/MIP  & \$.007899     & $ 1.0 \times   10^7 $  \\ \hline
	\end{tabular}
\end{table}

\end{frame}


%----------------------------------------------
\begin{frame}[plain]	
	\frametitle{Changing Roles of the OS}
	
	\begin{itemize}\Large
		\item OS depends on
		\begin{itemize}\large
			\item available hardware and software
			\item changing uses of machines
			\item changing expectations of users
			
		\end{itemize}
	\end{itemize}
	
	
\end{frame}

%----------------------------------------------
\begin{frame}[plain]	
	\frametitle{OS Concepts}
	
	\begin{itemize}\Large
		\item Single-Machine OS/VMM
		\begin{itemize}\large
			\item Memory management
			\item Process management
			\item Synchronization
			\item File systems and device support
			
		\end{itemize}
		\item Security
		\item Correctness
		\item Distributed System
	\end{itemize}
	
	
\end{frame}


%----------------------------------------------
\begin{frame}[t]
	\frametitle{Single-Machine OS/VMM}
	\begin{columns}[t]
		\begin{column}{.5\textwidth}
			
			\begin{itemize}\Large
				\item Purposes
				\begin{itemize}\large
					\item Clean virtual machine
					\item Hardware independence
					\item Resource sharing and management
					\item Persistent Data Storage
					\item Protection
					\item Real time support
					\item Parallelism
					\item human interface
					
				\end{itemize}
			\end{itemize}
			
		\end{column}
		
		\begin{column}{.5\textwidth}
			
			\begin{itemize}\Large
				\item Strategy
				\begin{itemize}\large
					\item How do we organize the OS
					effectively for development,
					evolution, performance,
					and security?
					\item How do we use multi-
					processor machines
					effectively?
					
				\end{itemize}
			\end{itemize}
			
		\end{column}
	\end{columns}
\end{frame}


%----------------------------------------------
\begin{frame}[t]
	\frametitle{Memory Management}
	\begin{columns}[t]
		\begin{column}{.5\textwidth}
			
			\begin{itemize}\Large
				\item Purposes
				\begin{itemize}\large
					\item Virtual memory: provides the
					illusion of infinite physical
					memory
					\item Swapping: moves processes to
					disk as necessary
					\item Paging: allows processes to run
					with only the active pages in
					memory
					\item Buffer Cache: speedup the IO
					access
					
				\end{itemize}
			\end{itemize}
			
		\end{column}
		
		\begin{column}{.5\textwidth}
			
			\begin{itemize}\Large
				\item Strategy
				\begin{itemize}\large
					\item How do we coordinate
					machines to share
					memory?
					\item How can we simplify
					memory management as
					memory becomes
					abundant?
					
				\end{itemize}
			\end{itemize}
			
		\end{column}
	\end{columns}
\end{frame}

%----------------------------------------------
\begin{frame}[t]
	\frametitle{Process Scheduling}
	\begin{columns}[t]
		\begin{column}{.5\textwidth}
			
			\begin{itemize}\Large
				\item Purposes
				\begin{itemize}\large
					\item Provides the illusion of multiple
					processes running at the same
					time on a single processor
					\item Context switching: changing the
					attention of the processor
					\begin{itemize}\large
						\item Involves saving and restoring
						states
						\item Necessary to cross kernel
						boundary
					\end{itemize}
				\end{itemize}
			\end{itemize}
			
		\end{column}
		
		\begin{column}{.5\textwidth}
			
			\begin{itemize}\Large
				\item Strategy
				\begin{itemize}\large
					\item How do we achieve fairness,
					high throughput, and
					responsiveness at the same
					time?
					\item How do we reduce or avoid the
					cost of context switching?
					
				\end{itemize}
			\end{itemize}
			
		\end{column}
	\end{columns}
\end{frame}

%----------------------------------------------
\begin{frame}[t]
	\frametitle{Synchronization }
	\begin{columns}[t]
		\begin{column}{.5\textwidth}
			
			\begin{itemize}\Large
				\item Purposes
				\begin{itemize}\large
					\item Provides correct
					execution or
					coordinating threads
					in the face of
					arbitrary context
					switching
					
				\end{itemize}
			\end{itemize}
			
		\end{column}
		
		\begin{column}{.5\textwidth}
			
			\begin{itemize}\Large
				\item Strategy
				\begin{itemize}\large
					\item Atomic actions: all or nothing
					\item Mutual exclusion: one thread in the
					critical section at a time
					\item Semaphores: atomic, counter-
					based locks
					\item Avoid Deadlock: circular waiting on
					resources
					
				\end{itemize}
			\end{itemize}
			
		\end{column}
	\end{columns}
\end{frame}

%----------------------------------------------
\begin{frame}[t]
	\frametitle{File Systems}
	\begin{columns}[t]
		\begin{column}{.5\textwidth}
			
			\begin{itemize}\Large
				\item Purposes
				\begin{itemize}\large
					\item File: data + attributes
					\item File system services:
					\begin{itemize}\large
						\item Organization
						\item Naming
						\item Access
						\item Synchronization
						\item Protection and security
	
					\end{itemize}					
				\end{itemize}
			\end{itemize}
			
		\end{column}
		
		\begin{column}{.5\textwidth}
			
			\begin{itemize}\Large
				\item Strategy
				\begin{itemize}\large
					\item How do we make different file
					systems work together, even across
					machines?
					\item How do we provide consistency,
					availability, and reliability to copies
					of a file across multiple machines?
					\item How do we handle very large data
					sets?
					
				\end{itemize}
			\end{itemize}
			
		\end{column}
	\end{columns}
\end{frame}


%----------------------------------------------
\begin{frame}[t]
	\frametitle{I/O Device }
	\begin{columns}[t]
		\begin{column}{.5\textwidth}
			
			\begin{itemize}\Large
				\item Purposes
				\begin{itemize}\large
					\item I/O devices tend to be a
					lot slower than memory
					speed
					\item Caching: stores extra
					data in memory in hope
					of near-term reuse
					
				\end{itemize}
			\end{itemize}
			
		\end{column}
		
		\begin{column}{.5\textwidth}
			
			\begin{itemize}\Large
				\item Strategy
				\begin{itemize}\large
					\item How do we coordinate
					the memory resources
					across machines to
					enhance performance?
					\item How do we handle new
					devices with new
					characteristics?
				\end{itemize}
			\end{itemize}
			
		\end{column}
	\end{columns}
\end{frame}
%----------------------------------------------
\end{document}