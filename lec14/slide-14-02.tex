\input{../preamble}

%----------------------------------------------------------------------------------------
%	TITLE PAGE
%----------------------------------------------------------------------------------------

\title[第14讲]{第14讲: Concurrency in OS Kernel} % The short title appears at the bottom of every slide, the full title is only on the title page
\subtitle{第二节:Scalable Concurrency -- Lock}
\author{陈渝} % Your name
\institute[清华大学] % Your institution as it will appear on the bottom of every slide, may be shorthand to save space
{
	清华大学计算机系 \\ % Your institution for the title page
	\medskip
	\textit{yuchen@tsinghua.edu.cn} % Your email address
}
\date{\today} % Date, can be changed to a custom dateC++ memory order



\begin{document}

\begin{frame}
\titlepage % Print the title page as the first slide
\end{frame}
    
%\begin{frame}
%\frametitle{提纲} % Table of contents slide, comment this block out to remove it
%\tableofcontents % Throughout your presentation, if you choose to use \section{} and \subsection{} commands, these will automatically be printed on this slide as an overview of your presentation
%\end{frame}
%
%%----------------------------------------------------------------------------------------
%%	PRESENTATION SLIDES
%%----------------------------------------------------------------------------------------
%

%----------------------------------------------
%-------------------------------------------------
\begin{frame}
    \frametitle{Resource}
    
    
    

	\begin{columns}
    
    \begin{column}{.5\textwidth}
        \centering
        
        \includegraphics[width=.7\textwidth]{book-cpp-concurrency-in-action}
        
    \end{column}
    
    \begin{column}{.5\textwidth}
        
      \includegraphics[width=.7\textwidth]{perfbook}
       
        
    \end{column}
    
    
\end{columns}
    
    \tiny Reference:
    
    "Is Parallel Programming Hard, And, If So, What Can You Do About It?",Paul McKenney;\\
    "C++Concurrency in Action", ANTHONY WILLIAMS; \\
    "CS510 - Advanced Topics in Concurrency", Jonathan Walpole;很 
    Adam Belay from MIT PDOS
\end{frame}

%----------------------------------------------
\begin{frame}[fragile]
    \frametitle{Scalable Locking}
    \Large
    \begin{itemize}
    \item Why do we need locking in the kernel?
    \item Which problems are we trying to solve?
    \item What implementation choices do we have?
    \item Is there a one-size-fits-all solution?
  
\end{itemize}
    
\end{frame}


%----------------------------------------------
\begin{frame}[fragile]
    \frametitle{ticket spinlock}
    \Large
    Goal
    \begin{itemize}
        \item Correctness:  Mutual exclusion, Progress, Bounded wait
        \item  Fairness       
        \item Performance
        
        
    \end{itemize}
    
\end{frame}


%----------------------------------------------
\begin{frame}[fragile]
    \frametitle{ticket spinlock}
    \Large
    Idea:
    \begin{itemize}
        \item reserve each thread's turn to use a lock
        \item each thread spins until their turn
        \item Use new atomic primitive:
        fetch-and-add (FAA)
        \item Spin while not thread's ticket != turn
        \item Release: Advance to next turn
        
 \end{itemize}
    
\end{frame}


%----------------------------------------------
\begin{frame}[fragile]
    \frametitle{ticket spinlock}
%    \large    
    \begin{block}{}
        \begin{verbatim}
typedef  struct {
    int ticket;
    int turn;
} lock_t;

void lock_init(lock_t *lock) {
    lock->ticket = 0;
    lock->turn = 0;
}
void acquire(lock_t *lock) {
    int myturn = FAA(&lock->ticket);
    while (lock->turn != myturn); // spin
}
void release(lock_t *lock) { lock->turn += 1; }

        \end{verbatim}
    \end{block} 
\end{frame}

%----------------------------------------------
\begin{frame}[fragile]
    \frametitle{ticket spinlock}
    \Large
    Ticket lock time analysis
    
    \begin{itemize}
        \item Atomic increment – O(1) broadcast message        
        \item Then read-only spin, no cost until next release             
        \item  release OP invalidates message sent to all cores, and O(N) find messages, as they re-read        
        \item  But fairness and less bus traffic while spinning
        
        
    \end{itemize}
    Ticket are “non-scalable” locks, cost of handoff scales with number of waiters
    
    
\end{frame}
%----------------------------------------------
\begin{frame}[fragile]
    \frametitle{MCS lock}
    \Large

    \begin{itemize}
        \item Goal: O(1) message release time
        \item Can we wake just one core at a time?
        \item  Idea: Have each core spin on a different cache-line
    \end{itemize}    
\end{frame}

%----------------------------------------------
\begin{frame}[fragile]
    \frametitle{MCS lock}
    \Large
    
    \begin{itemize}
        \item Each CPU has a qnode structure in its local memory

    \begin{block}{}
    \begin{verbatim}
    typedef struct qnode {
        struct qnode *next;
        bool locked;
    } qnode;
\end{verbatim}
\end{block}         
        \item  A lock is a qnode pointer to the tail of the list      
        \item While waiting, spin on local locked flag
        
    \end{itemize}    
\end{frame}

%----------------------------------------------
\begin{frame}[fragile]
    \frametitle{MCS lock}
\centering
\includegraphics[width=1.\textwidth]{mcs-lock}
\end{frame}



%----------------------------------------------
\begin{frame}[fragile]
    \frametitle{MCS lock}
    \Large
     Acquiring MCS locks
       
        \begin{block}{}
            \begin{verbatim}
acquire (qnode *L, qnode *I) {
    I->next = NULL;
    qnode *predecessor = I;
    XCHG (*L, predecessor);
    if (predecessor != NULL) {
        I->locked = true;
        predecessor->next = I;
        while (I->locked) ;
    }
}
\end{verbatim}
        \end{block}         

\end{frame}


%----------------------------------------------
\begin{frame}[fragile]
    \frametitle{MCS lock}
    \Large
    Releasing MCS locks
    
    \begin{block}{}
        \begin{verbatim}
    release (lock *L, qnode *I) {
        if (!I->next)
            if (CAS (*L, I, NULL))
                return;
        while (!I->next) ;
        I->next->locked = false;
    }
\end{verbatim}
    \end{block}         
    
\end{frame}


%----------------------------------------------
\begin{frame}[fragile]
    \frametitle{MCS lock}
    \centering
    Locking strategy comparison
    
    \includegraphics[width=.7\textwidth]{mcs-lock-perf}
\end{frame}

%----------------------------------------------
\begin{frame}[fragile]
    \frametitle{MCS lock}
    \centering
    But not a  panacea
    \includegraphics[width=.7\textwidth]{mcs-lock-perf2}
\end{frame}
%----------------------------------------------
%----------------------------------------------
\end{document}